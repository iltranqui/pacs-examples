\documentclass[10pt]{beamer}
\usetheme{default}
\setbeamercovered{invisible}
\setbeamertemplate{navigation symbols}{}
\setbeamertemplate{footline}{
    \flushright{\hfill \insertframenumber{}/\inserttotalframenumber}
}

\usepackage{listings}

% User-defined colors.
\definecolor{DarkGreen}{rgb}{0, .5, 0}
\definecolor{DarkBlue}{rgb}{0, 0, .5}
\definecolor{DarkRed}{rgb}{.5, 0, 0}
\definecolor{LightGray}{rgb}{.95, .95, .95}
\definecolor{White}{rgb}{1.0,1.0,1.0}
\definecolor{darkblue}{rgb}{0,0,0.9}
\definecolor{darkred}{rgb}{0.8,0,0}
\definecolor{darkgreen}{rgb}{0.0,0.85,0}

% Settings for listing class.
\lstset{
  language=C++,                        % The default language
  basicstyle=\small\ttfamily,          % The basic style
  backgroundcolor=\color{White},       % Set listing background
  keywordstyle=\color{DarkBlue}\bfseries, % Set keyword style
  commentstyle=\color{DarkGreen}\itshape, % Set comment style
  stringstyle=\color{DarkRed}, % Set string constant style
  extendedchars=true % Allow extended characters
  breaklines=true,
  basewidth={0.5em,0.4em},
  fontadjust=true,
  linewidth=\textwidth,
  breakatwhitespace=true,
  showstringspaces=false,
  lineskip=0ex, %  frame=single
}

\begin{document}
    \title{A complete Newton solver\protect\\ using \texttt{Eigen}}
    \author{Pasquale Claudio Africa}
    \date{08/05/2020}

\begin{frame}[plain, noframenumbering]
    \maketitle
\end{frame}

\begin{frame}{Newton solver}
This example (a simplified version of \texttt{Examples/src/NewtonSolver}) is about a set of tools that implement generic Newton or quasi-Newton methods to determine the zero of a system of non-linear equations, based on the \texttt{Eigen} library.

The code structure is the following:
\begin{itemize}
\item \texttt{NewtonTraits} contains the definition of the types used by the main classes, to guarantee uniformity.
\item \texttt{JacobianBase} is a base class which implements the action of a \textit{quasi-Jacobian}: the user may choose among \texttt{FullJacobian} where the actual Jacobian (or an approximation of it) must be specified by the user, and \texttt{DiscreteJacobian}, that approximates the Jacobian via finite differences.
\item \texttt{JacobianFactory} instantiates a concrete derived class of \texttt{JacobianBase} family on the fly.
\item \texttt{Newton} applies the Newton method, given the non-linear system and a \texttt{JacobianBase}.
\item \texttt{NewtonOptions} and \texttt{NewtonResults} encapsulate the input options and the output results.
\end{itemize}
\end{frame}

\end{document}
