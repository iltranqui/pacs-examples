\documentclass[10pt]{beamer}
\usetheme{default}
\setbeamercovered{invisible}
\setbeamertemplate{navigation symbols}{}
\setbeamertemplate{footline}{
    \flushright{\hfill \insertframenumber{}/\inserttotalframenumber}
}

\usepackage{listings}

% User-defined colors.
\definecolor{DarkGreen}{rgb}{0, .5, 0}
\definecolor{DarkBlue}{rgb}{0, 0, .5}
\definecolor{DarkRed}{rgb}{.5, 0, 0}
\definecolor{LightGray}{rgb}{.95, .95, .95}
\definecolor{White}{rgb}{1.0,1.0,1.0}
\definecolor{darkblue}{rgb}{0,0,0.9}
\definecolor{darkred}{rgb}{0.8,0,0}
\definecolor{darkgreen}{rgb}{0.0,0.85,0}

% Settings for listing class.
\lstset{
  language=C++,                        % The default language
  basicstyle=\small\ttfamily,          % The basic style
  backgroundcolor=\color{White},       % Set listing background
  keywordstyle=\color{DarkBlue}\bfseries, % Set keyword style
  commentstyle=\color{DarkGreen}\itshape, % Set comment style
  stringstyle=\color{DarkRed}, % Set string constant style
  extendedchars=true % Allow extended characters
  breaklines=true,
  basewidth={0.5em,0.4em},
  fontadjust=true,
  linewidth=\textwidth,
  breakatwhitespace=true,
  showstringspaces=false,
  lineskip=0ex, %  frame=single
}

\begin{document}
    \title{A Runge-Kutta-Fehlberg solver\protect\\ using traits and \texttt{Eigen}}
    \author{Pasquale Claudio Africa}
    \date{29/05/2020}

\begin{frame}[plain, noframenumbering]
    \maketitle
\end{frame}

\begin{frame}{RKF solver}
This example (a modified version of \texttt{Examples/src/RKFSolver}) is about a set of tools that implement embedded Runge-Kutta-Fehlberg methods to solve non-linear scalar, vector and matrix Ordinary Differential Equations, based on the \texttt{Eigen} library.
\vfill
Consider an ODE of the form
\begin{equation*}
\frac{\mathrm{d} y}{\mathrm{d} y} = f(t, y)
\end{equation*}
and consider the coefficients \(b_i, b_i^*, a_{ij}, c_i\) to be given in the form of a Butcher tableau.
We recall that, at each time step \(t_n\), an embedded RKF method of order \(p\) consists of the following steps.
\begin{enumerate}
\item Compute the high-order solution \(y_{n+1} = y_{n} + h \sum_{i=1}^p b_i k_i\), where \(k_i = f\left(t_n + c_i h_n, y_n + \sum_{j=1}^{i-1}a_{ij}k_j\right)\), and \(h_n = t_{n+1} - t_n\).
\item Compute the low-order solution \(y_{n+1}^* = y_{n} + h \sum_{i=1}^p b_i^* k_i\).
\item Compute the error \(\varepsilon_{n+1} = y_{n+1} - y_{n+1}^* = h\sum_{i=1}^p(b_i - b_i^*)k_i\).
\item Adapt the step size \(h_{n+1} = \tau_{n} h_{n}\), where \(\tau_{n}\) is a reduction (\(<1\))/expansion (\(>1\)) factor depending on whether \(\varepsilon_{n+1}\) is larger or smaller than a prescribed tolerance.
\end{enumerate}
\end{frame}

\begin{frame}{RKF solver}
The code structure is the following:
\begin{itemize}
\item \texttt{ButcherRKF} contains the definition of the Butcher tableaux for the most common RKF methods.
\item \texttt{RKFTraits} defines the basic structure that enable to statically select the type of the equation(s) to be solved (\textit{i.e.} scalar, vector, matrix).
\item \texttt{RKFResult} is a data structure containing the output of the RKF solver.
\item \texttt{RKF} implements a generic RKF solver interface, filling a proper \texttt{RKFResult} object.
\end{itemize}
\end{frame}

\begin{frame}{Homeworks}
\begin{enumerate}
\item Use the provided solver to solve the SIR model presented in \url{https://arxiv.org/pdf/2003.00122.pdf}.
\item Implement the Fehlberg12 and the Dormand-Prince methods (\url{https://en.wikipedia.org/wiki/List_of_Runge\%E2\%80\%93Kutta_methods\#Embedded_methods})
\item Modify the constructor of the \texttt{RKF} class so that the chosen RKF method has a default value.

\item Define a class to handle the input options for the \texttt{RKF} class, and provide the corresponding setter/getter methods and possibly a method to parse options from file.

\item (Advanced) Implement the factory pattern to the \texttt{ButcherArray} class, so that the actual method can be chosen dynamically exploiting polymorphism.
\end{enumerate}
\end{frame}

\end{document}
