\documentclass{article}
\usepackage{amsmath,amssymb}
\begin{document}
This document presents two examples where floating point round-off error may lead to completely wrong results. These are clearly degenerate, specially crafted,  cases. Yet, they should indicate you that floating point operations may need care.

\section{Rump example}
The example considers the function
\[
  r(x,y)=Ay^6+x^2(Bx^2y^2-y^6-Cy^4-D)+Ey^8+\frac{x}{Dy}
\]
 with
  $A=333.75$, $B=11$, $C=121$, $D=2$ and $E=5.5$.
  
  The function is very badly conditioned at $(x,y)=(77617,33096)$.
  Taking the first $6$ significant digits, we have
  $r(77617,33096)=-0.827396$ (you may verify it using a symbolic
  computation tool). However, the condition number, estimated by
  $\vert r^\prime\vert/\vert r\vert$, is of the order of $10^{32}$ at
  that point. Consequently, floating point errors may be amplified
  strongly, and indeed the computed value using floating point
  arithmetic differs from the exact one by orders of magnitude, even
  with the extended precision given by \texttt{long double}.

  The reason is that the powers of $x$ and $y$ causes the computed
  value to loose significant digits, and the cancellation error due to
  the subtraction is finally amplified. This is clearly a specially
  crafted example, however it shows the danger of loosing significant
  digits when computing powers. Interesting, the actual result may
  differ when running the code on different architectures, or
  compiling with different compiler options, because of possible differences
  on the implementation of floating point arithmetic, or different
  optimization steps carried out be the compiler.

  Unfortunately, in this example there is little we can do to
  ameliorate the situation. 
  
\section{Wrongly converging sequence}.
We define a sequence $x_n,\, n=0,1,\ldots$ as
\[
  \begin{array}{l}
    x_0=2,\\
    x_1=-4,\\
    x_{n+1}=111-1130x_n^{-1}+3000x^{-1}_{n-1}x^{-1}_{n},\ n=2,3,\ldots.
  \end{array}
\]
    It can be easily found, by looking at the roots of the polynomial
    \[
    x^3-111+1130x^2-3000,
    \]
    that if the sequence converges it converges necessarily to either
    $5$, $6$, or $100$.
    
    By performing the limit via symbolic computations, it may be
    assessed that indeed $\lim_{n\to\infty} x_n=6$.

    However, we now consider the fixed point problem governed by the
    iteration function $\phi:\,\mathbb{R}^2\to\mathbb{R}^2$ given by
    \[
      \begin{bmatrix}
        y_1\\y_2 \end{bmatrix}=\mathbf{y}=\phi(\mathbf{y})=\begin{bmatrix}
        y_2\\
        111-1130y^{-1}_1+3000y^{-1}_{1}y^{-1}_{2}
        \end{bmatrix}
    \]
    It can be noted that the iterate
    $\mathbf{y}_{n+1}=\phi(\mathbf{y}_{n}),\, n=0,\ldots$, with
    $\mathbf{y}_0=[2,-4]^T$, provides values that corresponds to the
    given sequence, more precisely $x_n=y_{2,n-1}$ for $n>2$. Indeed,
    the limit of the given sequence (in exact arithmetic) corresponds
    to the fixed point $(6,6)=\phi(6,6)$.
    
    It may be verified that also $(5,5)$ and $(100,100)$ are fixed
    points of $\phi$. However, we have that
    $\Vert\nabla\phi(6,6)\Vert>>1$, while
    $\Vert\nabla\phi(100,100)\Vert<1$ (and
    $\Vert\nabla\phi(5,5)\Vert>>1$). Thus, even if $6$ is the exact
    limit point of the given sequence, the iteration function $\phi$
    in $(6,6)$ is not a contraction. It means that in general the
    fixed point iteration $\mathbf{y}_{n+1}=\phi(\mathbf{y}_n)$, for a
    given initial condition $\mathbf{y}_0$, is \emph{not convergent} to that
    point.  In \emph{exact arithmetic} it may converge to $(6,6)$ only
    for special isolated values of the initial condition, for instance
    the one given by $\mathbf{y}_0=(2,-4)$.  While, $\phi$ is a
    contraction in the neighborhood of $(100,100)$, thus there is a
    whole interval of possible initial values for which the fixed
    point iteration converges to $(100,100)$.

    As a consequence of these considerations, $6$ is an unstable limit
    point, in the sense that any small perturbations will lead the
    iteration to move away from it and finally towards the ``stable''
    point $100$. The ``perturbation'' is, on our case, induced by
    round-off errors. And, indeed, the sequence computed numerically
    in floating point arithmetic eventually converges to $100$, even
    when using extended precision.

    Running the given code, you may note that in the first few
    iterations the sequence is in fact approaching $6$. However, as
    soon as round-off error accumulates, the iterate enters the basin
    of attraction of $x=100$ and very rapidly converges to that
    (wrong) value. In extended precision, from iteration $35$ we get
    $100$ with all significant digits.

    If you change in the code \texttt{long double} with
    \texttt{double} in the template argument of
    \texttt{Pathologic\_Sequence}, you may note that the deviation
    from the convergence to $6$ appears earlier, because of the larger
    round-off errors.
 
\end{document}
%%% Local Variables:
%%% mode: latex
%%% TeX-master: t
%%% End:
